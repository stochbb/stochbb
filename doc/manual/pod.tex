\section{Divergence point example}
This example shows a simple \emph{toy model} for some cognitive process that is able to produce a
so called \emph{divergence point} \cite{Reingold2012}. A divergence point of two response-latency
distributions is the earliest time point at which the two response-latency distributions differ. Obviously, 
there exists no divergence point, if the two response latency distributions are analytic 
(e.g., the convolution of Gamma distributions, \cite{Gelooven1999}). This example, however,
not only shows how a non-analytic response-latency distribution may arise that allows for a 
divergence point, but actually has a divergence point%
\footnote{At least one distribution being non-analytic is a necessary but not a sufficient
condition for the existence of a divergence point.}. 

This very simple model consists of 3 Gamma-distributed random variables $C, X_1$ and $X_2$,
where the latter is a shifted Gamma distribution (here shifted by $300$ ms, i.e. 
$X_3 \sim \Gamma(T-300; k, \theta)$).  The model is
\begin{align}
 \text{control: } & R_c = C + X_1 & \text{ where } C \sim \Gamma(c; 3, 20),\,X_1\sim\Gamma(x_1; 10,30)\\
 \text{experimental: } & R_e = C + \min(X_1, X_2) & \text{ where } X_2 \sim\Gamma(x_2-300; 1, 70)\,.
\end{align}

This model can be read as: Under control condition, the stimulus triggers a common stage $C$ which then 
triggers a second stage $X_2$ that itself immediately triggers the response. Under experimental condition,
again the stimulus triggers the common stage $C$ which then triggers $X_1$. Additionally, the common
stage $C$ also triggers the delayed stage $X_2$ in parallel to $X_1$. The response is then triggered by 
either $X_1$ or $X_2$, depending on which stage completes first. Consequently, the response latency
under control condition is $R_e = C + \min(X_1,X_2)$.

The following code shows how this simple model is implemented using the Python API of StochBB.

\begin{lstlisting}[language=Python]
from numpy import *
import stochbb;

# Processing model
d = 300
C = stochbb.gamma(3,20)
X1 = stochbb.gamma(10,30)
# shifted exponential
X2 = stochbb.gamma(1,70) + d

# response latency control condition
#   is simply R = C + X1
Rc = C + X1
# and for the experimental condition
Re = C + stochbb.minimum(X1, X2)

# Eval PDF & CDF
Tmin, Tmax, N = 0, 1200, 1200;
Tc = empty((N,)); Rc.density().eval(Tmin, Tmax, Tc)
Te = empty((N,)); Re.density().eval(Tmin, Tmax, Te)
TCc = empty((N,)); Rc.density().evalCDF(Tmin, Tmax, TCc)
TCe = empty((N,)); Re.density().evalCDF(Tmin, Tmax, TCe)
\end{lstlisting}

\begin{figure} [!ht]
 \centering
 \includegraphics[width=.6\textwidth]{pod.pdf}
 \caption{PDFs (upper panel) and survival functions (lower panel) of the response latencies under 
 control condition (blue lines) and experimental condition (green lines).  The divergence point of 
 the two response-latency distributions is shown as the red vertical lines. \label{fig:pod}}
\end{figure}

Figure \ref{fig:pod} shows the plots generated from the code above. The blue lines show the PDF
(upper panel) and survival function (1-CDF, lower panel) of the response latency under control
condition. Being a convolution of two Gamma distributions, it is an analytic function on the
interval $(0,\infty)$ \citep{Gelooven1999}. The green lines show the PDF and survival function
of the response latency under control condition. As the distribution of $X_2$ is not analytic on the complete interval
$(0,\infty)$ (discontinuity at $T=300$ ms), a divergence point may exist at $T=300$ (red vertical
lines).  In fact, that point is visible in both the PDF and the survival function. Hence, this simple
model is able to produce a divergence point as both response-latency distributions are identical
on the interval $[0,300)$ and diverge thereafter.
