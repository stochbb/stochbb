\section{An example: The divergence point}
This example shows a simple \emph{toy model} for some cognitive process that is able to produce a
so called \emph{divergence point} \cite[e.g.,][]{Reingold2012}. A divergence point of two response-latency
distributions is the earliest time point at which the two distributions differ. Obviously, 
there exists no divergence point, if the two distributions are analytic 
(e.g., the convolution of Gamma distributions, \cite{Gelooven1999}). This example, however,
not only shows how a non-analytic response-latency distribution may arise using very common
processing stages models (i.e., these stage models are also used in EZ-Reader, \cite{Reichle2003}) 
that allows for a divergence point, but actually has a divergence point%
\footnote{At least one distribution being non-analytic is a necessary but not a sufficient
condition for the existence of a divergence point.}. 

This very simple model consists of 4 Gamma-distributed stages $V, S_1, S_2$ and $M$,
where the second stage $S_2$ is delayed by $500$ ms. 
The model is
\begin{align}
 \text{control: } & R_c = V + S_1 + M \\
 \text{experimental: } & R_e = V + \min(S_1, 500+S_2) + M\,,
\end{align}
where $V \sim \Gamma(?, ?),\,S_1\sim\Gamma(?,?)\,M\sim\Gamma(?,?)$ and $X_2 \sim\Gamma(?, ?)$.
\begin{figure}[!ht]
 \centering
 \includegraphics[width=0.7\textwidth]{example2.png}
 \caption{Network of the second example model as visualized by \pkg{StochBB}.} \label{fig:example2}
\end{figure}

This model (also shown in Figure \ref{fig:example2}) can be read as: Under control condition, the stimulus triggers a common \emph{visual} 
stage $C$ which then triggers a second stage $S_1$ that itself immediately triggers the response stage $M$. Under experimental condition,
again the stimulus triggers the common stage $V$ which then triggers $S_1$. Additionally to the sage $S_1$, the common
stage $V$ also triggers the delayed stage $S_2$ in parallel to $X_1$. The response stage $M$ is then triggered by 
either $X_1$ or $X_2$, depending on which stage completes first. Consequently, the response latency
under experimental condition is $R_e = V + \min(S_1,S_2+500) + M$.

\begin{figure} [!ht]
 \begin{subfigure}[t]{0.45\textwidth}
   \includegraphics[width=.95\textwidth]{example2_plot.pdf}
   \subcaption{PDFs of the response latencies under  control condition (blue lines) and experimental condition (green lines).  
   The \emph{true} divergence point of the two response-latency distributions is $500 ms$, however, the distributions appear 
   to diverge much later (about $600 ms$). \label{fig:pod}}
  \end{subfigure} \hfill
 \begin{subfigure}[t]{0.45\textwidth}
   \includegraphics[width=.95\textwidth]{example2_zoom_plot.pdf}
   \subcaption{The same PDFs but zoomed in. Here it get visible that the \emph{true} PD is much earlier that visible in the right plot. In fact the \emph{true} DP is not visible under any zoom level, as the PDFs diverge very slowly.} \label{fig:podzoom}
  \end{subfigure}
  \caption{Comparison of the response latency distributions as obtained from the mode shown in Fig. \ref{fig:example2}.}
  \label{fig:podplots}
\end{figure}

Figure \ref{fig:podplots} shows the plots generated by \pkg{StochBB}. The blue lines show the PDF
of the response latency under control condition. Being a convolution of 3 Gamma distributions, it 
is an analytic function on the interval $(0,\infty)$ \citep{Gelooven1999}. The red lines show the 
PDF of the response latency under control condition. As the distribution of $S_2$ is not analytic on 
the complete interval $(0,\infty)$ (discontinuity at $T=500$ ms), a divergence point may exist at 
$T=500$. Due to the \emph{minimum} stage and the fact that $P(S_2<500)=0$ (delayed by 
$500 ms$), it is ensured that the two response latencies are identical on the interval $[0,500) ms$.
Hence, this simple model is able to produce a divergence point as both response-latency distributions are identical
on the interval $[0,500)$ and diverge thereafter.

However, Fig. \ref{fig:podzoom} shows that this existing divergence point cannot be estimated reliably as the two 
response latency distributions diverge very slowly. Consequently, an estimate for this point (e.g. by means of 
estimators proposed in \citep{Reingold2012}) based on a finite sample will always be heavily biased towards larger
values.