\section{Artificial example} \label{sec:example}
Within this section, I present how a rather trivial stochastic model of a reading experiment can be
implemented using StochBB's GUI application.

The model describes the fixation duration $R$ as a simple chain of a lexical $L$, semantic $S$ and
occulomotor $M$ stage. Each stage is represented by a Gamma distributed random variable, that is
\begin{align*}
 L &\sim \Gamma(5\,f+5, 10)\\
 S &\sim \Gamma(10\,p+5, 20)\\
 M &\sim \Gamma(10, 30)\\
 \Rightarrow R &= L+S+M\,,
\end{align*} 
where $f$ is the words log-frequency and $p$ its predictability. In a reading experiment, 
participants usually do not read a single word with a fixed frequency and predictability but
rather read several words with different frequencies and predictabilities. Hence it is necessary
to model the predictabilities and frequencies as random variables too, for example
\begin{align*}
 f &\sim U[0,4]\\
 p &\sim U[0,1]\,.
\end{align*}
With this, the lexical ($L$) and semantic ($S$) stages are not Gamma distributed anymore but
compound distributed random variables. 

\begin{figure}[!ht]
 \centering
  \includegraphics[width=0.7\textwidth]{example1.png}
  \caption{Visualization of the example process within the \pkg{StochBB} GUI application.} \label{fig:exgraph}
\end{figure}

Figure \ref{fig:exgraph} shows the visualization of the network within the \pkg{SochBB} GUI application. At the top-left corner an additional node called \emph{Stimulus} is shown. This node represents a simple $\delta$-distributed random variable defining the temporal reference point at $T=0$. The random parameters $f$ and $p$ as well a their affine transformation ($5\,f+5$ and $10\,p+5$) are shown in the bottom left of Fig. \ref{fig:exgraph}. The compound Gamma distributed $L$ and $S$ stage as well as the Gamma distributed $M$ stage are shown near the center. At the very right of Fig. \ref{fig:exgraph}, a plot node is shown that produces the Figure \ref{fig:example}.

\begin{figure}[!ht]
 \centering
 \includegraphics[width=0.45\textwidth]{example1_plot.pdf}
 \caption{Shows the histogram of the samples (boxes) and the numerically obtained densities (lines).} \label{fig:example}
\end{figure}	

Figure \ref{fig:example} shows the results of the analysis. It shows the densities of the response times for each stage (L, S and M) in comparison. 
