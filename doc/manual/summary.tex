\section{Summary}
With this article I introduced \pkg{StochBB}, a framework that allows to analyze systems of
dependent random variables efficiently without resorting to stochastic simulation. It uses
analytic solutions for the marginal distributions where possible and resorts to numerical
approximations if necessary. It provides a surprising good precision, one that would require a
large amount of samples using stochastic simulation to achieve same precision. To this end, it is
possible to fit networks of random variables to data by means of maximum likelihood or minimizing
the KS-statistic.

Additionally, \pkg{StochBB} provides a GUI, which eases assembling of complex networks that are
mend to describe random models like those typical for cognitive psychology by providing an abstract
semantics related to \emph{stochastic processing stages} rather than \emph{random variables}. The
abstract network representation of the GUI application eases the analyses of complex networks such
that undergraduate students in applied fields are able to analyze and simulate such networks
without the need to \emph{translate} them to the \emph{random variable} representation. To this
end, the \pkg{SochBB} is a suitable tool to complement undergraduate lectures.

The software (including the core library, interfaces to \proglang{Python} and \proglang{R} as well
as the GUI application) is available under the General Public License version 3 at 
\href{https://github.com/stochbb}{https://github.com/stochbb}.
